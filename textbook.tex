\documentclass[
  11pt,
,
onecolumn,
openany
]{book}

% Standard Packages %
\usepackage[colorlinks]{hyperref}
\usepackage{longtable,booktabs,array}


% figures %
\usepackage{graphicx}
% Redefine \includegraphics so that, unless explicit options are
% given, the image width will not exceed the width or the height of the page.
% Images get their normal width if they fit onto the page, but
% are scaled down if they would overflow the margins.
\makeatletter
\def\ScaleWidthIfNeeded{%
 \ifdim\Gin@nat@width>\linewidth
    \linewidth
  \else
    \Gin@nat@width
  \fi
}
\def\ScaleHeightIfNeeded{%
  \ifdim\Gin@nat@height>0.9\textheight
    0.9\textheight
  \else
    \Gin@nat@width
  \fi
}
\makeatother

\setkeys{Gin}{width=\ScaleWidthIfNeeded,height=\ScaleHeightIfNeeded,keepaspectratio}%

% Math Packages %
\usepackage{amsmath,amssymb}
\usepackage{ntheorem}

% Load mdframed package for styling content blocks %
\usepackage{xcolor}
\usepackage[framemethod=default,backgroundcolor=lightgray]{mdframed}

% Summary Content Block Styles %
\newenvironment{learning-objectives}
{\begin{mdframed}}
{\end{mdframed}}

% Pandoc LaTeX Configurations %
\providecommand{\subtitle}[1]{% add subtitle to \maketitle
  \apptocmd{\@title}{\par {\large #1 \par}}{}{}
}

\providecommand{\tightlist}{%
  \setlength{\itemsep}{0pt}\setlength{\parskip}{0pt}}

    \setcounter{secnumdepth}{-\maxdimen} % remove section numbering

% Code Block Shading %

% Load Fonts %
\usepackage[osf,p]{libertinus}

% Textbook Metadata %

\title{Dissertation Prospectux}
\author{María Palacio \and Dr.~Lucille Kerr}
\date{}

% CSL References %

\begin{document}


\maketitle

\pagestyle{empty}

%% copyright page

\begingroup
\footnotesize
\parindent 0pt
\parskip \baselineskip

Prospectus for dissertation project

\textcopyright{}  María Palacio \and Dr.~Lucille Kerr \\
All rights reserved.

This work may be distributed and/or modified under the conditions
of the true License.

\begin{center}
 99 32 11 88 48 01\hspace{2em}9 9 8 6 5 4 %1 
\end{center}

\begin{center}
\begin{tabular}{ll}
First edition:  & May 2013 \\
Second impression, with minor extensions & January 2009 \\
Third impression, with minor extensions & May 2013 
\end{tabular}
\end{center}

\vfill

Catalogging in publication data


\vfill

Publishing Organization, \\
Evanston, IL \\
\texttt{https://www.library.northwestern.edu}

%%%%{\LARGE\plogo}
\vspace*{2\baselineskip}


\endgroup
\clearpage


\renewcommand*\contentsname{Contents}
\setcounter{tocdepth}{2}
\tableofcontents



\hypertarget{dissertation-prospectus}{%
\chapter{Dissertation Prospectus}\label{dissertation-prospectus}}

\hypertarget{that-town-is-a-red-zone-geographies-of-the-colombian-armed-conflict-in-21st-century-testimonial-narratives}{%
\section{\texorpdfstring{``That town is a red zone'': Geographies of the
Colombian Armed Conflict in 21\textsuperscript{st}-Century Testimonial
Narratives}{``That town is a red zone'': Geographies of the Colombian Armed Conflict in 21st-Century Testimonial Narratives}}\label{that-town-is-a-red-zone-geographies-of-the-colombian-armed-conflict-in-21st-century-testimonial-narratives}}

I. Research Questions

The term ``zonas rojas'' (red zones), which has pervaded State, media, and
cultural discourses since the 1990s, refers to the places in which the
Colombian armed conflict has been most prevalent. What is the historical and
conceptual background that gives the term ``zona roja'' its current meaning in
Colombian contemporary testimonial narratives? How do these narratives
represent the relationship between such ``red'' spaces and their citizens? How
will understanding the pervasiveness and development of the term ``zona roja''
contribute to the study of the testimonial genre within the Colombian context
and beyond? In what ways do Colombian testimonial narratives remap, redefine
and reformulate the area of Latin American Testimonial Studies?

\hypertarget{i.-premise}{%
\chapter{I. Premise}\label{i.-premise}}

My research interrogates the configuration and representation of the concept
of the ``zona roja,'' and how Colombian 21\textsuperscript{st}-century
testimonial narratives represent these zones' territorial realities. The
Colombian armed conflict has a long and complex history dating from the
bipartisan civil war known as ``La Violencia'' (1925-1958). Today, the armed
conflict continues, but its agents are no longer the two original and distinct
factions (\emph{liberales} y \emph{conservadores}) motivated by opposing
political ideologies.\footnote{A possible definition of ``map'' is Jeremy
  Black's in \emph{Visions of the World: A History of Maps}: ``One definition
  that simply will not do, therefore, is that a map is a representation of
  reality. It is fairer to argue that it is a depiction of spatial
  relationships. This definition accepts the range of `mapped' as the variety
  of ways in which such relationships can be represented'' (Black 10).}
Furthermore, as the country's agricultural and economic centers have shifted,
the agricultural frontiers have also moved, along with the conflict's
geographic axis. Thus, the conflict's participants, settings, and intensity
have changed throughout its unfolding history. One thing, however, remains the
same: the Colombian armed conflict takes place predominantly on the fringes of
rural and urban areas (i. e., in populated areas where State control is in
dispute). Most victims of the conflict therefore come from communities in
rural areas and urban outskirts that have been classified by the State, at
different times, as ``\emph{Zonas Especiales de Orden Público}'' (Law
Enforcement Special Zones) or ``\emph{Zonas Estratégicas de Intervención
Integral}'' (Strategic Zones of Comprehensive Intervention) -- or, as they are
more commonly known, ``zonas rojas'' (red rones).\footnote{``La Violencia''
  was a time of armed confrontation that took place mainly in rural areas,
  between members of the liberal and conservative parties. Although typically
  understood as a conflict between partisan ideologies, this phase of the
  struggle was essentially a conflict related to land distribution and
  ownership. During ``La Violencia'' several peasants' self-defense groups of
  liberal ideology emerged. These groups were the precursors of communist
  guerrillas formed during Frente Nacional.} This representation of the
``zonas rojas'' homogenizes the characterizations and problems of the
territories and populations most impacted by the conflict, perpetuating the
idea that war is the same everywhere, undifferentiated by local context.
Moreover, the pervasive usage of this term fuels the discourse that claims
that red zones lack State presence, and that both the territories and their
inhabitants are irremediably ``violent,'' ``disorganized,'' and
``uncivilized.''\footnote{The Frente Nacional (National Front) was a political
  pact between elites from both liberal and conservative parties that aimed to
  put an end to bipartisan violence. During the Frente Nacional, the
  government promoted the pacification of the rural areas affected by ``La
  Violencia.'' However, the government did not introduce land reform to
  resolve the problems of land distribution and ownership, so several of the
  liberal self-defense groups that originated during ``La Violencia,''
  continued to exist, were persecuted by the State, and later became liberal
  and communist guerrillas.}

The armed conflict has not been foreign to representation. In fact,
historians, journalists, sociologists, and victims have documented and
fictionalized the struggle in historical novels, articles, chronicles, and
\emph{testimonios}. To date, studies on such texts have focused, with good
reason, on how these accounts represent, document, and divulge the facts of
what occurred during the conflict. These studies about contemporary
testimonial narratives allow us to have a better perspective of the armed
conflict in Colombia, and the regions in which it has taken place. Beginning
from an understanding of space as a socially and discursively constructed
entity (Lefebvre, \emph{The Production of Space}), I intend to examine a group
of testimonial narratives about war in Colombia published in the last twenty
years, emphasizing their portrayal of space and how the armed conflict and its
representations have affected the relationship between certain territories and
their inhabitants. My goal is to question the uses, meanings and pervasiveness
of the term ``zona roja'' in Colombia, and to contest the importance of the
term in shaping the views held by the general population about regions
affected by the armed conflict. Especially, I aim to analyze changes in
testimonial narratives between the first and the second decade of twenty-first
century: how these works localize and represent the spaces where armed
conflict has taken place. I intend to examine the political, social, and
cultural factors that may have contributed to these changes and whether
readers' understanding of the so-called ``zonas rojas'' have been influenced
by these testimonial accounts.

\hypertarget{iii.-rationale}{%
\chapter{III. Rationale}\label{iii.-rationale}}

As a result of the two peace agreements with illegal armed groups over the
last few decades, \emph{memory} and \emph{victim} have become central
categories in historical and political discussions, as well as in the
production of testimonial works at the beginning of the 21st century. One
might say that Colombia is witnessing a ``memory boom'' in which attention is
focused on uncovering the truth about the perpetrators and causes of the armed
conflict, and on giving the victims a platform for narrating their
experiences. Given the ample testimonial production from these recent decades
and the new voices and perspectives that have emerged from the scenes of the
memory of conflict, it is not surprising that literary criticism has begun to
focus attention on contemporary Colombian testimonial works. Two recent
projects stand out for the ways in which they approach contemporary
narratives: the aforementioned Eduardo Suárez Gómez's \emph{La literatura
testimonial como memoria de las guerras en Colombia} (2016), and María Ospina
Pizano's \emph{El rompecabezas de la memoria. Literatura, cine y testimonio de
comienzos de siglo en Colombia} (2019). Suárez Gómez focuses on two works that
represent two different modalities of testimonial discourse: mediated
testimony and direct testimony. He studies how these works take as a point of
reference the collective memory of previous conflicts in Colombia to make
sense of the contemporary experiences. Ospina Pizano considers a larger corpus
and suggests interpreting recent testimonial production as more than
representations of the armed conflict or records of contemporary history.
Instead, she claims that testimonial discourses from the 21st century are
interventions in the ways of interpreting the violence of recent decades that
do more than merely representing the conflict: they contribute to the
construction of a collective memory.

These two critical studies, among others, focus on the construction of memory
about the generally traumatic experiences caused by the armed conflict by
making emphasis on the events of the conflict. Given that space is a
fundamental aspect of how people experience events, and of how they recall,
represent, and make sense of those events, studying the representation and
discursive construction of spaces where the conflict took place becomes a
fundamental part of reconstructing the memory of the Colombian armed conflict.
My project aims to insert itself within the ongoing ``memory boom''
conversation in Colombia, emphasizing a key aspect of Colombian testimonial
narratives that has not yet been addressed: the representation, discursive
construction, and territorial realities of the geographies of the conflict.

Furthermore, I intend to ``locate'' Colombian testimonial narratives on the
``map'' of Latin American testimonial studies, which so far have focused
attention primarily on testimonial traditions from Central America and the
Southern Cone. Central American studies on testimony have revolved around
subalterns' representation and self-representation, as well as their role in
civil wars during the 1980s. In the Southern Cone, critics have studied mainly
how testimonies reconstruct the memory of events during the military
dictatorships in Argentina, Chile, and Uruguay. Due to the length of the
Colombian armed conflict, and its diverse causes and agents, testimonial
narratives have tried to account for the realities and spaces in which there
is a convergence of several historicities, witnesses, and interpretations.
Studies on testimonial narratives must, therefore, pay closer attention to the
specificity of Colombian testimonial narratives and how they have responded to
the task of representing, remembering, and interpreting such complex
realities. A conversation between Colombian and Latin American testimonial
narratives is essential, not only to study how the former have shaped the
latter, but also because the Colombian case allows us to question the limits,
scope, and possibilities of the testimonial genre as a whole.

\hypertarget{ii.-scope}{%
\chapter{II. Scope}\label{ii.-scope}}

\hypertarget{a.-testimonial-scope}{%
\section{A. Testimonial Scope}\label{a.-testimonial-scope}}

This project's primary sources belong to the field of testimonial discourse. I
define testimonial narratives as cultural products that bear witness to an
event or real-life experience through a narration. Although the primary
textual sources I will study in my dissertation differ in terms of production
and reception, they all represent witnesses who account for experiences
related to events within the context of the Colombian armed conflict. By
bearing witness to such experiences, these texts presuppose a link between the
testimonial account and the facts. Therefore, one cannot study these
testimonial narratives without considering how testimonial discourses produce
a reality effect, how they relate to a real-life referent, and what kind of
mediations exist between the account of an experience and the experience
itself. While early on in discussions of testimonial narrative critics such as
John Beverley (``The Margin at the Center,'' 1989) considered as ``genuine
\emph{tesitmonios}'' the works written or edited by a professional writer,
narrated in the voice of a subaltern subject who was an actor in, and a
witness to, the life experiences or events to which the text testifies
(Beverley 32), other scholars such as Elzbieta Sklodowska (``Hacia una
tipología del testimonio latinoamericano,'' 1990) suggested that testimonial
discourse has different modalities and that works can be categorized by
placing them in a range that goes from the most fictional-least referential to
the least fictional-most referential (Sklodowska 109). My selection of
testimonial materials has followed Sklodowska's proposal: the dissertation's
corpus comprises diverse modalities of Colombian testimonial discourse, which
I have classified under five categories.

1. Testimonial fictions: works of fiction that refer to historical events or
real places. For example, María Ospina Pizano's short story titled
``Policarpa'' (2015) --which fictionalizes the writing process of a former
guerrilla woman's testimony-- or Daniel Ferreira's \emph{La rebelión de los
oficios inútiles} (2019) --which, narrated as if it were a piece of
investigative journalism, tells the story of a fictional 1970s peasants'
collective that occupies a vacant lot in the outskirts of a small town in
Colombia.

2. Testimonial novels: works that represent real events and places,
emphasizing their factuality. For example, in \emph{Río muerto} (2020) by
Ricardo Silva Romero, the author-narrator reconstructs the last days of the
Palacios family in Belén de Chamí after the father is murdered by a
paramilitary group, and presents the story as based on the testimony of a
survivor of the Palacios family, who told the author-narrator his story while
sitting in Bogotá's heavy traffic. \emph{El ruido de las cosas al caer}
(2012), by Juan Gabriel Vázquez, is a testimonial novel that reconstructs the
story of the birth of the drug trafficking business in Medellín.

3. Testimonios: texts written or edited by a professional writer, whose
narrator is a person who witnessed and took part in the events and experiences
told in the text. For example, \emph{Ahí le dejo esos fierros} (2009) gathers
six stories written and edited by Alfredo Molano, who gives voice to
demobilized militants of different armed groups who bore first-hand witness to
the complexity of Colombian conflict. \emph{Los niños de la guerra. Quince
años después} (2015) by Guillermo González Uribe is a compilation of stories
that testify to the scars left by the armed conflicts on the children and
teenagers of Colombia.

4. Autobiographic testimonies: works in which the author-narrator was a
witness to the narrated events. For example, Mery Yolanda Sánchez's \emph{El
atajo} (2014) and \emph{Cómo mate a mi padre} by Sara Jaramillo Klinkert
(2019). In \emph{El atajo,} Sánchez narrates, in the first person, her
experience as cultural promoter in Nariño, a region besieged by violence. In
\emph{Cómo maté a mi padre} Jaramillo Klinkert recounts her and her family's
life experiences after the father was killed by a hitman in Medellín in the
1990s.

5. Journalistic chronicles: works based on documentation and journalistic
research. For example, \emph{Rutas del conflict} is an investigative
journalism website dedicated to tracing the effects of the conflict in
different regions of Colombia. The podcast \emph{Relatos anfibios} presents a
group of journalists' narrations of the stories and experiences of people
living in different regions of Colombia, especially in those regions most
affected by the war.

\hypertarget{b.-conceptual-scope}{%
\section{B. Conceptual Scope}\label{b.-conceptual-scope}}

The concept ``zona roja'' (red zone) is part of the discursive apparatus
employed by the Colombian State to control its territory. By classifying a
geographic space as a red zone or conflict zone, the State decides what
resources to invest in those territories. Over time, however, this
classificatory term has permeated most discourses about the armed conflict and
the nation, from media to popular culture and fiction. In \emph{The Production
of Space}, Henri Lefebvre claims that space is socially and discursively
constructed-\/-that is, space is not just the physical setting where human
interactions and conflicts occur, but a symbolic entity subject to disputes
over its interpretation and representation (Lefevbre 38). Both the concept of
``zona roja,'' and the spaces considered as such in Colombia, have been
subject to both real and symbolic disputes over their control and
interpretation. In this regard, testimonial narratives --which I define as
discursive productions that give voice to witnesses' experiences-- promote, at
least in theory, discourses that may challenge spatial configurations imposed
by the State, including the notion of ``zona roja.''

Mapping is a method states use to represent, shape, and control their
territories. More than accurate representations of reality, maps are
interpretations of the relations between spaces, territories, and populations
that are produced by an individual or a society.\footnote{A possible
  definition of ``map'' is Jeremy Black's in \emph{Visions of the World: A
  History of Maps}: ``One definition that simply will not do, therefore, is
  that a map is a representation of reality. It is fairer to argue that it is
  a depiction of spatial relationships. This definition accepts the range of
  `mapped' as the variety of ways in which such relationships can be
  represented'' (Black 10).} Devising a map requires studying, generalizing,
homogenizing, and hierarchically organizing a territory and its inhabitants;
it requires a selective operation that highlights the aspects of a territory
that its author wants to emphasize. Maps can therefore be read, not only as
state control tools, but as purposely crafted visual discourses about a
territory and its inhabitants. Likewise, the narration of an event involves
creating a context; it is a selective exercise that highlights the relevant
aspects of a reality and thereby embeds it with meaning. I propose to read
primary testimonial works as discursive maps, so to speak, as representations
of real events that also interpret the relations between spaces, places, and
people within conflict zones, and that shape and are shaped by the notion of
``zonas rojas.''

To account for the way in which testimonial narratives ``map'' red zones, I
propose three concepts to guide my analysis of the representation and
construction of spaces: delocalization, hyperlocation, and displacement.
During the first decade of the twenty-first century, several novels were
published with the conflict zones as their main setting. Among those novels,
two were critically acclaimed both nationally and internationally: Héctor Abad
Faciolince's \emph{Angosta} (2005) is a metaphorical spatial depiction of
Colombian reality and Evelio José Rosero's \emph{Los ejércitos} (2006) tells
the story of a man from a rural area who witnesses how his town --a fictional
place located in an undetermined place within Colombian territory-- gets
caught in the crossfire between several armed groups. Both novels invent
fictional spaces that can be read as metaphors of conflict zones in Colombia.
By delocalizing conflict zones --that is, by placing them in unspecified
spaces in Colombia, or refraining from giving them a name that connects them
to a real place, these narratives do more than preserve memory of the armed
conflict; indeed, they contribute to the homogenization of the ``zonas
rojas.''

In recent years, however, testimonial works (journalistic chronicles,
documentaries, and testimonial fictions), have shifted to a hyperlocation of
the spaces where the narrated events of conflict take place. Instead of
setting stories in metaphorical or unnamed places, these narratives constantly
name, describe, and locate in detail the spaces where events occur, and
sometimes provide geographical location coordinates. Both Phillip Potdevin's
\emph{La sembradora de cuerpos} and Juan Miguel Álvarez's collection of
articles titled \emph{Cuadernos de los encuentros. Verde tierra calcinada}
--published in 2019-\/-are examples of works that ``hyperlocate'' the ``red
zones.'' Finally, in the context of the Colombian armed conflict, many people
have had to leave their homelands to escape violence. Forced displacement has,
therefore, inevitably shaped the way in which space is represented,
remembered, and imagined in a significant number of testimonial narratives.
For example, Alfredo Molano's \emph{Desterrados, crónicas del desarraigo} and
the collection \emph{Yo sobreviví'' Memorias de guerra y resistencia en
Colombia} (2018), by the team of Rutas del Conflicto, talk about the space
from a position of exile or displacement-\/-that is, they talk about a space
that no longer exists, or that is no longer what it used to be, and therefore
cannot be recovered.

\hypertarget{c.-historical-scope}{%
\section{C. Historical Scope}\label{c.-historical-scope}}

Historians, sociologists, and political scientists continue to debate about
the precise timeline for the history of the Colombian armed conflict. Despite
the lack of a consensus about the timeline, I believe that the conflict can be
organized into four main periods: the phase of bipartisan violence (La
Violencia bipartidista) (1925-1958),\footnote{``La Violencia'' was a time of
  armed confrontation that took place mainly in rural areas, between members
  of the liberal and conservative parties. Although typically understood as a
  conflict between partisan ideologies, this phase of the struggle was
  essentially a conflict related to land distribution and ownership. During
  ``La Violencia'' several peasants' self-defense groups of liberal ideology
  emerged. These groups were the precursors of communist guerrillas formed
  during Frente Nacional.} the Frente Nacional (1958-1974),\footnote{The
  Frente Nacional (National Front) was a political pact between elites from
  both liberal and conservative parties that aimed to put an end to bipartisan
  violence. During the Frente Nacional, the government promoted the
  pacification of the rural areas affected by ``La Violencia.'' However, the
  government did not introduce land reform to resolve the problems of land
  distribution and ownership, so several of the liberal self-defense groups
  that originated during ``La Violencia,'' continued to exist, were persecuted
  by the State, and later became liberal and communist guerrillas.} the
emergence of drug and paramilitary violence (1980s and 1990s), and the Plan
Colombia period (from 2001 to the present day). This periodization coincides
partially with the phases suggested by Carlos Miguel Ortiz (``Historiografía
de la violencia,'' 1994) and Jorge Eduardo Suárez Gómez (\emph{La literatura
testimonial como memoria de las guerras en Colombia}, 2016) regarding all
written testimonial materials about the Colombian armed conflict. I intend to
focus on testimonial narratives that narrate events that have taken place
during the two last periods of the above timeline. I have selected this time
frame for my reseach for two reasons: first, the armed conflict grew more
complex after the 1980s due to the incorporation of new participants and
sources of funding; second, in the last two decades the government and several
illegal armed groups reached peace agreements that, albeit imperfect, brought
about a ``memory boom'' that has significantly increased the production of
narratives about the comtemporary armed conflict.

Drug cartel and paramilitary violence emerged between the 1980s and 1990s. In
this period, the increase in production and distribution of cocaine completely
altered the scene of war in Colombia, diversifying both the types of violence
and its agents. Initially funded by powerful drug lords and landowners,
diverse paramilitary groups emerged with the initial goal of confronting the
communist guerrillas. The emergence of these groups intensified violence in
areas traditionally controlled by guerrillas. In this period there were
battles not only between the National Army and guerrillas, but also between
the guerrillas and paramilitary groups fighting over control of the territory.
Furthermore, during this period the peasant massacres increased greatly:
paramilitary groups terrorized civilians as a means to gain and maintain
control over guerrilla-influenced areas. Moreover, both guerrillas and
paramilitary groups began to fund themselves with drug money, and drug-related
violence spread both in rural areas and in the main cities.

In 1987 the government appointed a commission to study ``La Violencia,'' which
resulted in the publication of \emph{Colombia, violencia y democracia.} Partly
thanks to this research, public understanding of the Colombian conflict
underwent a transformation: originally seen as an essentially political
problem-\/-or a confrontation between left and right-\/-the conflict came to
be understood as a multi-causal and multi-modal confrontation. Testimonial
narratives played a key role in shifting public perception of the conflict as
they ``addressed stories of violence different from political violence''
(Suárez Gómez 58; my trans). Precipitated by the diversity of causes and types
of violence within the Colombian armed conflict, testimonial texts from this
period portrayed the experiences of individuals from all social classes (60).
As in other parts of Latin America, the mediation of intellectuals
(journalists, sociologists, and writers) became essential to the testimonial
narratives from this period, but, unlike most testimonial narratives in other
Latin American countries, whose paradigm was Rigoberta Menchú's \emph{Me llamo
Rigoberta Menchú y así me nació la conciencia} (1983), Colombian testimonial
narratives did not only emphasize political struggle and the experience of
subaltern subjects.

In 1999, the Andrés Pastrana administration began peace dialogues between the
government and Farc guerrillas, which eventually broke down in
2001.\footnote{Peace talks between the Pastrana administration and Farc took
  place in the so-called ``Zona de distension'' (demilitarized zone), an area
  of about 16000 square miles between the provinces of Meta and Caquetá. The
  government demilitarized this area so that the peace talks could be
  conducted safely. In practice, Farc took advantage of the situation by
  strengthening its power over the territory and expanding its drug
  trafficking routes. In the collective imagination this region therefore
  became one of the ``zonas rojas'' par excellence.} Along with the
negotiations with Farc, President Pastrana negotiated an economic and military
aid plan with the United States-\/-the Plan Colombia-\/-to combat drug
trafficking and, by extension, Colombian guerrillas. When the peace talks
failed, the Pastrana government, and that of his successor Álvaro Uribe Vélez,
implemented strategies to defeat the insurgency militarily, funded in great
part by the US thanks to Plan Colombia. Uribe based his government on the
pillar of Seguridad Democrática (democratic security), a policy that
strengthened the presence of the State in areas traditionally controlled by
guerrillas through the militarization of those territories. Although this
policy hurt the guerrillas, especially the Farc, it also increased violence
against civilians in certain regions of the country.

During his administration, Uribe negotiated the demobilization of paramilitary
groups in exchange for lower sentences for its members.\footnote{In 2003, the
  AUC-\/-the Colombian United Self-Defense group (that at the time gathered
  paramilitary groups from different regions of the country)-\/- signed the
  Ralito Accord with the Uribe administration. As a result, 18,000 AUC members
  demobilized. The Ralito Agreement was harshly criticized as it gave too many
  concessions to the paramilitaries without demanding reparation for their
  victims in return. In 2005, Colombian Congress passed the Justicia y Paz law
  (justice and peace), which provided a legal framework for the demobilization
  of paramilitaries who were supposed to testify before the Justice and Peace
  Courts. Under that law, another 30,000 paramilitaries demobilized. However,
  years later paramilitary leaders confessed to the Justice and Peace Courts
  that some of the 2003 mobilizations were a sham.} In 2012, Juan Manuel
Santos, who succeeded Uribe in the presidency, began new negotiations to reach
a peace accord with the Farc. These negotiations, which took place between
2012 and 2016, resulted in the \emph{Acuerdo final para la terminación del
conflicto y la construcción de una paz estable y duradera} (General Accord to
End the Conflict and Build a Stable, Enduring Peace). The implementation of
the accord began with the demobilization of all the Farc guerrilla fronts and
the creation of their political party --initially named Farc, and now the
Comunes party. However, this accord did not put an end to conflict: there was
a power vacuum that followed the Farc demobilization, which, together with the
State's failure to fulfill the accord, led to intensified violence in
territories that enjoyed a period of peace.

Suárez Gómez characterizes Colombian testimonial literature produced at the
beginning of the 21\textsuperscript{st} century as an explosion of stories
about drug trafficking and kidnappings, rather than about guerilla and
paramilitary violence. Although he rightly points out that publishing houses
promoted ``instant books'' of former hostages during the first years of the
21\textsuperscript{st} century, testimonial narratives from this period didn't
deal exclusively with kidnappings. The 2006 and 2016 peace accords with,
respectively, the paramilitaries and Farc changed how the nation dealt with
the memory of the armed conflict, as the public debate began reconsidering
guerrilla and paramilitary violence. Alongside the peace accords there have
been State initiatives that aim to reveal what really happened during the
conflict and to give a voice to its victims. At the same time, independent
journalistic projects have taken an interest in telling the stories of the
regions most affected by guerrilla, State, and paramilitary violence.
Furthermore, testimonial fictions are addressing anew events that occurred
during the last decades of the 20th and the first decades of the
21\textsuperscript{st} centuries and have thereby diversified testimonial
cultural production about the conflict.

\hypertarget{iv.-structure}{%
\chapter{IV. Structure}\label{iv.-structure}}

\textbf{A. Introduction}

\begin{quote}
1. Historical context. Definition of conflict zones.
\end{quote}

2. 20\textsuperscript{th} and 21\textsuperscript{st} century testimonial
narratives in Colombia.

3. Space as a social and discursive production.

\textbf{B. Chapter 1: Delocalization: Invented ``Red'' Towns}

1. \emph{Angosta} (2005), by Hector Abad Faciolince, as metaphorical ``zona
roja''

2. The invention of a town in the middle of the war in \emph{Los ejércitos}
(2006) by Evelio Rosero

3. The unnamed/un-namable spaces of \emph{Los estratos} (2013) by Juan
Cárdenas~

\textbf{C. Chapter 2: Displacement: Witnesses of the No-place}

1. Journey to the center of conflict in testimonial narratives: \emph{El
atajo} (2014) y \emph{Los niños de la guerra. Quince años después} (2016)~

2. The combatant's movement in \emph{testimonios} and testimonial fictions:
\emph{Ahí le dejo esos fieros} (2009) y ``Policarpa'' (2015)

3. Landless farmers, forced displacement, and the \emph{testimonios} of the
exile: \emph{Desterrados, crónicas del desarraigo} (2001) y ``\emph{Yo
sobreviví'' Memorias de guerra y resistencia en Colombia} (2018)~

\textbf{D. Chapter 3: Hyperlocation: Testimonial Territories}

1. On site journalism: \emph{Rutas del conflicto} y \emph{Cuaderno de los
encuentros. Verde tierra calcinada} (2019)\emph{~}

2. Territory and historical memory: Reports from Centro Nacional de Memoria
Histórica (National Center of Historical Memory)

3. Hyperlocated testimonial fictions: \emph{La sembradora de cuerpos} (2019)
and \emph{Río muerto} (2020)

\textbf{E. Conclusion}


\end{document}